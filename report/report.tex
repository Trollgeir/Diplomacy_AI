\documentclass[a4paper]{article} % a4paper option recommended
\usepackage[english]{babel}
\usepackage{uureport}
\usepackage{cite}
\usepackage{hyperref}
\bibliographystyle{alpha}

\title{Diplomacy}
\type{Report}
\course{Games \& Agents}

\author{A. Berland, B. Reigersberg, E. Koens, \\J. de Groot, K. van Katwijk, T. de Goey}

\begin{document}
\maketitle
\tableofcontents
\newpage

\section{Introduction}
Diplomacy is an online multiplayer game based on the board game invented in 1954 by Allan B. Calhamer. The online version of the game has 137966 subscribed players, it can however be played by bots as well. A few bots have already been implemented, so far the best bot on the market is Albert. All these bots lack one common property, which is the ability to asses the other players. Therefore we tried to implement an Artificial Intelligent bot for this game, which is able to comprehend the actions of other bots and players. Our bot can deduce the best possible moves which are based on his own ideas and knowledge, he will also concern his current belief on the other player or bot. We made quite some progress and we will elaborate on our development of this bot in this paper. example of a citation \cite{dipblue}. 

\section{Gains \& Weights}

In order for our AI to move units around the map, we make use of a gains \& weights system. For every province we calculate a gain value. This values signifies the importance of this province based on a variety of properties and variables. This gain is weighed by the weight value, which not only signifies the likeliness of succes to take this province - but also the cost (for example, do we defect allies with this action?).

\subsection{Gains}
As you win the game when you control the majority of supply centers, it is important for our AI to move it's units toward supply centers. Therefore we have chosen to base the gain value of a province on wether it is a supply center or not. A province without a supply center has no direct gain. A province with a supply center has a gain based on the type of supply center it is. 

Some special supply centers are home supply centers. These are home and starting supply center to a specific power. A power can only build new units on his home supply centers.   

In order to assign a gain, we distinguish the following types of supply centers: 

\begin{description} 

\item[Normal Supply centers]
are all supply centers, excluding home supply centers, not under our control (So either neutral/untaken or under control by another power). As these supply centers are not under our control, we can potentially take them in order to bring us closer to victory. These supply centers should have a relatively high gain value. 

\item[Home supply centers]
are all home supply centers, excluding our own. Taking such a supply center does not just take us one step closer to victory, but also effectively disables a power from building units there. To consider this, we will give this type a slightly higher gain than normal supply centers only if it is under control by it's original power. If for example Austria is controlling Russia's home supply center, Russia is already unable to build units there. In this case there should not be an extra gain and it is treated as a normal supply center.    

\item[Our supply centers] are all supply centers, excluding \textit{our} home supply centers, under our control. As we already control these supply centers and want units to take other supply centers, we assign a lower gain to this type. However, in the case that this supply center is threatened by enemy units we will increase the gain to prevent losing control over this supply center.  

\item[Our home supply centers] are all of our home supply centers. As with other supply centers under our control, these will have a lower gain to prevent units from staying here, and will increase in case of a threat. The difference is that whenever this supply centers is under enemy control, it should have a very high gain as it is very important to take it back in order to build new units.   

\subsection{Smoothed gains}
We want units to move towards high gain provinces. Therefore provinces adjacent to high gain provinces should also have same gain - as moving to this province will move us towards a province with high gain. In order to achieve this we smooth the gains over the map after the gains have been calculated. For each province the smoothed gain $g_s$ is calculated as follows:

We tried: 

$$ g_{s} = g_0 + \frac{ C_1 \sum_i g_1^i}{n}$$

But right now we do :

$$ g_{s} = g_0 + C_1 \sum_i g_1^i$$

where $g_0$ is the gain of the province, $C_1$ is a constant, $g_1$ is a first-order neighbor of the province and $n$ is the number of provinces affecting the sum. 

\subsection{Threat}


\end{description} 

% \begin{thebibliography}{1}

%   \bibitem{notes} Pacheco, J.M., Santos, F.C., Souza, M.O. and Skyrms, B., {\em Evolutionary dynamics of collective action in N-person stag hunt dilemmas}  2009.

%   \bibitem{impj}  da Costa Ferreira, A.F., {\em DipBlue: a Diplomacy Agent with Strategic and Trust Reasoning} For Jury Evaluation.

% \end{thebibliography}

\bibliography{references}{}

\end{document}
